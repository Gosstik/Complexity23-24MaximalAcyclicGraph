\addcontentsline{toc}{section}{Литература}

\renewcommand\refname{Литература}
% % В самом списке 1. вместо [1]
% \makeatletter
% \renewcommand{\@biblabel}[1]{#1.}
\makeatother

{\small

  \begin{thebibliography} {21}

    \bibitem{cite1}
    N. Alon and J.H. Spencer, \textit{The Probabilistic Method} (John Wiley and Sons, New York, 1992).

    \bibitem{cite2}
    B. Berger and P.W. Shor, Approximation algorithms for the maximum acyclic subgraph problem, in: \textit{Proc.
      1st Ann. ACM-SIAM Symp. on Discrete Algorithms} (1990) 236-243 .

    \bibitem{cite3}
    M.M. Flood, Exact and heuristic algorithms for the weighted feedback arc set problem: A special case of the skew-symmetric quadratic assignment problem, \textit{Networks} 20 (1990) 1-23.

    \bibitem{cite4}
    A. Frank, How to make a digraph strongly connected, \textit{Combinatorica} 1 (1981) 145-153.

    \bibitem{cite5}
    H.N. Gabow, A representation for crossing set families with applications to submodular flow problems, \textit{Proc. 4th Ann. ACM-SIAM Symp. on Discrete Algorithms} (1993) 202-211.

    \bibitem{cite6}
    H.N. Gabow, A framework for cost-scaling algorithms for submodular flow problems, 1993.

    \bibitem{cite7}
    M. Grötschel, M. Jünger and G. Reinhelt, On the acyclic subgraph polytope, \textit{Mathematical Programming} 33 (1985) 28-42.

    \bibitem{cite8}
    M. Grötchel, L. Lovász and A. Schrijver, \textit{Geometric Algorithms and Combinatorial Optimization} (Springer, Berlin, 1988).

    \bibitem{cite9}
    B. Korte, Approximation algorithms for discrete optimization problems, \textit{Ann. Discrete Math.} 4 (1979) 85-120.

    \bibitem{cite10} R. Kaas, A branch and bound algorithm for the acyclic subgraph problem, \textit{European J. Oper. Res.} 8 (1981) 335-362.

    \bibitem{cite11} R.M. Karp, Reducibility among combinatorial problems, in: R.E. Miller and J.W. Thatcher, eds.,
    \textit{Complexity of Computer Computations} (Plenum, New
    York, 1972) 85-103.

    \bibitem{cite12} A. Karazanov, On the minimal number of arcs of a digraph meeting all its directed cutsets, abstract, \textit{Graph Theory Newsletters} 8 (1979).

    \bibitem{cite13}
    B. Korte and D. Hausmann, An analysis of the greedy heuristic for independence systems, \textit{Ann. Discrete Math.} 2 (1978) 65-74.

    \bibitem{cite14}
    M. Jünger, \textit{Polyhedral Combinatorics and the Acyclic Subgraph Problem} (Heldermann, 1985).

    \bibitem{cite15}
    C.L. Lucchesi, A minimax equality for directed graphs, Ph.D. Dissertation, University of Waterloo, Waterloo, Ontario, 1976.

    \bibitem{cite16}
    L. Lin and S. Sahni, Fair edge deletion problems, \textit{IEEE Trans. Comput.} 38 7(1989) 56-761.

    \bibitem{cite17}
    S. Miyano, Systematized approaches to the complexity of subgraph problems, \textit{J. Inform. Process.} 13 (1990) 442-447.

    \bibitem{cite18}
    M, Penn and Z. Nutov, Minimum feedback are set and maximum integral dicycle packing in $K_{3,3}$-free digraphs, 1993.

    \bibitem{cite19}
    C.H. Papadimitriou and M. Yannakakis, Optimization, approximation, and complexity classes, \textit{J. Comput, System Sci.} 43 (1991) 425-440 .

    \bibitem{cite20}
    V. Ramachandran, Finding a minimum feedback arc set in reducible flow graphs, \textit{J. Algorithms} 9 (1988)
    299-313.

    \bibitem{cite21}
    M. Yannakakis, Node- and edge-deletion $\mathsf{NP}$ complete problems, in: \textit{Proc. 10th ACM Symp. on Theory of Computing} (ACM, New York, 1978) 253-264.

  \end{thebibliography}

}